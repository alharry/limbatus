\documentclass[]{article}
\usepackage{lmodern}
\usepackage{setspace}
\setstretch{2}
\usepackage{amssymb,amsmath}
\usepackage{ifxetex,ifluatex}
\usepackage{fixltx2e} % provides \textsubscript
\ifnum 0\ifxetex 1\fi\ifluatex 1\fi=0 % if pdftex
  \usepackage[T1]{fontenc}
  \usepackage[utf8]{inputenc}
\else % if luatex or xelatex
  \ifxetex
    \usepackage{mathspec}
  \else
    \usepackage{fontspec}
  \fi
  \defaultfontfeatures{Ligatures=TeX,Scale=MatchLowercase}
\fi
% use upquote if available, for straight quotes in verbatim environments
\IfFileExists{upquote.sty}{\usepackage{upquote}}{}
% use microtype if available
\IfFileExists{microtype.sty}{%
\usepackage{microtype}
\UseMicrotypeSet[protrusion]{basicmath} % disable protrusion for tt fonts
}{}
\usepackage[margin=1in]{geometry}
\usepackage{hyperref}
\hypersetup{unicode=true,
            pdftitle={Life history of the common blacktip shark, Carcharhinus limbatus, from central eastern Australia and comparative demography of a cryptic shark complex},
            pdfborder={0 0 0},
            breaklinks=true}
\urlstyle{same}  % don't use monospace font for urls
\usepackage{graphicx,grffile}
\makeatletter
\def\maxwidth{\ifdim\Gin@nat@width>\linewidth\linewidth\else\Gin@nat@width\fi}
\def\maxheight{\ifdim\Gin@nat@height>\textheight\textheight\else\Gin@nat@height\fi}
\makeatother
% Scale images if necessary, so that they will not overflow the page
% margins by default, and it is still possible to overwrite the defaults
% using explicit options in \includegraphics[width, height, ...]{}
\setkeys{Gin}{width=\maxwidth,height=\maxheight,keepaspectratio}
\IfFileExists{parskip.sty}{%
\usepackage{parskip}
}{% else
\setlength{\parindent}{0pt}
\setlength{\parskip}{6pt plus 2pt minus 1pt}
}
\setlength{\emergencystretch}{3em}  % prevent overfull lines
\providecommand{\tightlist}{%
  \setlength{\itemsep}{0pt}\setlength{\parskip}{0pt}}
\setcounter{secnumdepth}{0}
% Redefines (sub)paragraphs to behave more like sections
\ifx\paragraph\undefined\else
\let\oldparagraph\paragraph
\renewcommand{\paragraph}[1]{\oldparagraph{#1}\mbox{}}
\fi
\ifx\subparagraph\undefined\else
\let\oldsubparagraph\subparagraph
\renewcommand{\subparagraph}[1]{\oldsubparagraph{#1}\mbox{}}
\fi

%%% Use protect on footnotes to avoid problems with footnotes in titles
\let\rmarkdownfootnote\footnote%
\def\footnote{\protect\rmarkdownfootnote}

%%% Change title format to be more compact
\usepackage{titling}

% Create subtitle command for use in maketitle
\newcommand{\subtitle}[1]{
  \posttitle{
    \begin{center}\large#1\end{center}
    }
}

\setlength{\droptitle}{-2em}

  \title{Life history of the common blacktip shark, \emph{Carcharhinus limbatus},
from central eastern Australia and comparative demography of a cryptic
shark complex}
    \pretitle{\vspace{\droptitle}\centering\huge}
  \posttitle{\par}
    \author{Alastair V Harry\(^{A,B,G}\), Paul A Butcher\(^C\), William G
Macbeth\(^D\), Jess AT Morgan\(^E\), Stephen M Taylor\(^B\), Pascal T
Geraghty\(^F\)\\
~\\
\(^A\)Centre for Sustainable Tropical Fisheries \& Aquaculture, College
of Marine \& Environmental Sciences, James Cook University, Townsville,
QLD, 4811, Australia\\
\(^B\)Department of Primary Industries and Regional Development, PO Box
20, North Beach, WA, 6920, Australia\\
\(^C\)Department of Primary Industries, New South Wales Fisheries, PO
Box 4321, Coffs Harbour, NSW 2450, Australia\\
\(^D\)FERM Services, PO Box 337, Sheffield, TAS, 7306, Australia\\
\(^E\)Molecular Fisheries Laboratory, School of Biomedical Sciences, The
University of Queensland, St.~Lucia, Queensland 4072, Australia\\
\(^F\)Roads and Maritime Services, 33 James Craig Road, Rozelle Bay,
NSW, 2039, Australia\\
\(^G\)Corresponding author:
\href{mailto:alastair.harry@gmail.com}{\nolinkurl{alastair.harry@gmail.com}}\\
\emph{Running title: Biology and comparative demography of Carcharhinus
limbatus}}
    \preauthor{\centering\large\emph}
  \postauthor{\par}
    \date{}
    \predate{}\postdate{}
  
\fontsize{12}{20}
\usepackage{fancyhdr}
\pagestyle{fancy}
\fancyhf{}
\fancyfoot[L]{\thepage}
\renewcommand{\headrulewidth}{0pt}
\renewcommand{\footrulewidth}{0pt}

\begin{document}
\maketitle

\newpage

\hypertarget{abstract}{%
\subsubsection{Abstract}\label{abstract}}

Common and Australian blacktip sharks (\emph{Carcharhinus limbatus} and
\emph{Carcharhinus tilstoni}) occur sympatrically in Australia where
they are reported as a complex due to their morphological similarities.
This study provides the first description of the life history of
\emph{C. limbatus} using samples from central eastern Australia, where
\emph{C. tilstoni} is rare. Females (68 - 267 cm TL, \emph{n} = 183) and
males (65 - 255 cm TL, \emph{n} = 292) both matured at 8.3 years and 200
cm TL, which exceeds the maximum length of \emph{C. tilstoni}. Vertebral
ageing revealed that female and male \emph{C. limbatus} lived to 22 and
24 years, respectively, exceeding known longevity in \emph{C. tilstoni}.
The intrinsic rate of population increase, \emph{r}, calculated using a
Euler-Lotka demographic method was 0.11 (0.02) yr\(^{-1}\) for \emph{C.
limbatus}, compared to 0.17 (0.02) and 0.20 (0.03) yr\(^{-1}\) for two
\emph{C. tilstoni} stocks. Despite their similar appearance, these
species differed in both their biological productivity and
susceptibility to fishing activities. Monitoring of relative abundance
should be a priority given they are likely to have divergent responses
to fishing.

\textbf{Additional keywords:} age and growth, reproductive biology,
Chondrichthyes, fisheries management

\newpage

\hypertarget{introduction}{%
\subsection{Introduction}\label{introduction}}

Cryptic species complexes pose many challenges for fisheries management
and biodiversity conservation. These are particularly evident in the
morphologically conservative class of chondrichthyan fishes where a
taxonomic rennaissance in recent years has revealed high cryptic
biodiversity (Marshall \emph{et al.} 2009; Ebert \emph{et al.} 2010;
White and Last 2012; Quattro \emph{et al.} 2013). By virtue, the
discovery of a complex violates the assumption that the species is a
single unit for management purposes, often with important consequences.
For example, declines in the Critically Endangered common skate, already
one of the best documented examples of near-extinction of a marine fish
(Brander 1981; Dulvy \emph{et al.} 2000), might have masked greater
declines in the larger and more intrinsically susceptible flapper skate,
\emph{Dipturus intermedius} (Iglésias \emph{et al.} 2010). Similarly, it
is likely that the California butterfly ray, \emph{Gymnura marmorata},
has a more restricted and fragmented range than previously thought as a
result of historical misidentification with the morphologically similar
and sympatric Mazatlan butterfly ray, \emph{Gymnura crebripuncata}
(Smith \emph{et al.} 2009). If hybridisation is occuring, the
productivity of a fish stock can also be reduced if hybrid fitness is
low and hybridisation common (Hatfield and Schluter 1999).

Australia's blacktip shark complex is another case that highlights the
difficulties associated with morphologically similar, co-occurring
sharks and rays. Study of this species complex over the past four
decades has left a confusing body of scientific literature that
obstructs effective management. The complex consists of two species; the
Australian blacktip, \emph{Carcharhinus tilstoni}, a shark endemic to
inshore coastal waters of northern Australia, and the more widely
distributed common blacktip, \emph{Carcharhinus limbatus}, a shark found
throughout inshore tropical and subtropical waters globally. Whitley's
(1950) initial account of \emph{C. tilstoni} described the shark as a
new species, but stated that it differed only from \emph{C. limbatus} in
its dentition. Subsequent taxonomic studies concluded it was a synonym
of \emph{C. limbatus}, a species that varies in many aspects of its
morphology across its range (Garrick 1982; Compagno 1988).

When Australia's northern shark resources developed during the 1970s,
\emph{C. limbatus} was initially thought to be the largest single
component of the catch in a Taiwanese surface gillnet fishery that
retained over 20 000 tonnes per year of pelagic fish and sharks from the
Arafura and Timor Seas (Millington and Walter 1981; Lyle 1984). The
proclamation of the Australian Fishing Zone brought this fishery under
Australian jurisdiction in 1979 and research surveys revealed that the
main target shark species was actually comprised of two distinct groups
of individuals separable by clasper lengths, vertebral counts, and
pelvic fin colouration (Stevens and Wiley 1986). The smaller maturing,
and far more common of these two groups was resurrected as \emph{C.
tilstoni} (Stevens and Wiley 1986). The other, much rarer group, which
had dark pelvic fin markings and was seemingly present in only
negligible quantities (a 1 to 300 ratio), was confirmed genetically to
be the true \emph{C. limbatus} (Lavery and Shaklee 1991).

The perception that \emph{Carcharhinus limbatus} was a minor component
of shark landings continued until the late 2000s (Last and Stevens 2009)
when a survey using mitochondrial DNA (mtDNA) sequences unexpectedly
found it to be present in quantities equal to \emph{C. tilsoni} (Ovenden
\emph{et al.} 2010). At around the same time, another study using mtDNA
taken from samples from temperate latitudes on the east coast recorded a
southward range expansion in excess of 1000 km for \emph{C. tilstoni},
challenging the perception that the species was confined to the tropics
(Boomer \emph{et al.} 2010). However, the validity of these findings was
brought into doubt when a mismatch between vertebral counts and mtDNA
species identity was found in Queensland blacktips (Harry \emph{et al.}
2012). This mismatch extended to various other morphological and
ecological characteristics of both species, including body size,
maturity, length at birth, and timing of birth. Investigation of nuclear
markers revealed this discordance to be the result of hybridisation
between the two species (Morgan \emph{et al.} 2012). More recent
research indicates that hybrids are introgressed and likely the result
of ancestral interbreeding between the two species (J.A.T. Morgan,
unpublished data).

Despite several decades of research on blacktip sharks in Australia,
considerable uncertainty remains over the range of both species, aspects
of their ecology, the impact of historical overfishing (Walters and
Buckworth 1997; Field \emph{et al.} 2012), and their relative abundances
and ratios in the multiple commercial fisheries in which they are
captured (Harry \emph{et al.} 2011; Tillett \emph{et al.} 2012). One
major obstacle to progress has recently been resolved with the discovery
that pelvic fin markings can be used to discriminate between the two
species with 90\% accuracy (Johnson \emph{et al.} 2017). Another major
issue that remains unaddressed is the absence of baseline life history
data for \emph{C. limbatus} from Australian waters. An understanding of
its life history is needed to determine the relative productivity of
this species and its suceptibility to fishing. It is also required to
help provide a clear picture of the ecological role this species plays.
It has generally been accepted that the larger \emph{C. limbatus} is
likely to be more intrinsically susceptible to fishing than \emph{C.
tilsoni} (Bradshaw \emph{et al.} 2013). However, with no practical way
to distinguish between them in catches until now and limited knowledge
of the biology of \emph{C. limbatus}, they are reported together, and
effectively managed as a combined taxonomic category in all fisheries
that catch them.

This study provides the first detailed account of the life history of
\emph{C. limbatus} in Australian waters. It is based on data obtained
from multiple sources off the central east coast of Australia, where
\emph{C. tilstoni} is largely or completely absent. The size, growth,
maturity, and reproductive biology are compared with those of \emph{C.
tilstoni} from previous studies in the northeastern state of Queensland
(Qld) (Harry \emph{et al.} 2013) and the Northern Territory (NT)
(Stevens and Wiley 1986; Davenport and Stevens 1988). Using demographic
analyses we examine differences in stock productivity and resilience to
fishing of sympatric blacktip sharks and discuss the management
implications for this species complex.

\hypertarget{materials-and-methods}{%
\subsection{Materials and Methods}\label{materials-and-methods}}

\hypertarget{reconstruction-of-c.-limbatus-life-history}{%
\subsubsection{\texorpdfstring{Reconstruction of \emph{C. limbatus} life
history}{Reconstruction of C. limbatus life history}}\label{reconstruction-of-c.-limbatus-life-history}}

The life history of \emph{C. limbatus} was reconstructed using samples
synthesised from four predominantly fishery-dependent sources from
southeast Qld and northern New South Wales (NSW) (Fig. 1). These were:
two observer surveys of the NSW Ocean Trap \& Line Fishery (OTLF)
carried out between 2008 and 2010 (Macbeth \emph{et al.} 2009) and in
2013 (Broadhurst \emph{et al.} 2014); a 2004 to 2007 study of the shark
fauna of Moreton Bay involving both fishery-independent sampling and a
commercial fisher from the East Coast Inshore Finfish Fishery (ECIFF)
(Taylor and Bennett 2013); and a sample of neonate sharks purchased from
the same fisher in 2007 used to count pre-caudal vertebrae (Harry
\emph{et al.} 2012).

Both the OTLF and ECIFF are multi-sector, multi-gear fisheries, and a
description of their characteristics is given in each of the respective
studies above. In summary, samples from the OTLF were predominantly
obtained from vessels targeting large sharks (\textgreater{} 2 m total
length) for their fins using demersal setlines, whereby multiple hooks
were attached to a horizontally-set, weighted ground line in waters 5 to
250 m depth (Macbeth \emph{et al.} 2009). Sampling occurred in NSW
coastal waters between \(28^\circ 04'S\) (just north of Tweed Heads) and
\(34^\circ 03'S\) (Sydney). Samples from the ECIFF were obtained from a
commercial vessel targeting smaller sharks (\textless{} 1.5 m) and
teleosts for their flesh in shallow (\textless{} 2 m depth), estuarine
habitats of Moreton Bay using 700 - 800 m of small-mesh (7.6 - 15.2 cm)
gillnet.

Total length (TL) in cm was measured by placing the shark ventral side
down with the upper lobe of the caudal fin depressed in line with the
body axis. All further references to length throughout are in TL or have
been converted to TL (see Supplementary Material). Body mass was
measured to the nearest kilogram in sharks from NSW and nearest 0.1 kg
in those from Qld. Sex was determined based on the presence of claspers
on males, and outer clasper length (CL) was measured to the nearest mm
between the point of the pelvic fin insertion and the tip of the clasper
(Compagno 1984). In Qld the presence of an unhealed umbilical scar was
also recorded to determine size and timing of birth. A section of 3 to 5
vertebrae was sampled from the cervical region of each vertebral column
for age determination.

Sampling techniques were comparable for each of the four data sources,
although due to the fishery-dependent nature of most sampling it was
seldom possible to measure all variables for each individual. In
particular, vertebrae were only collected from the first OTLF survey
from 2008 to 2010, while weight data were primarily collected during the
second OTLF survey in 2013.

The species identity of all individuals sampled from NSW between 2008
and 2010 (Fig. 1) was determined using mitochondrial (mtDNA) and nuclear
(nDNA) genotyping (Morgan \emph{et al.} 2011, 2012). Although Taylor
\emph{et al.} (2013) did not confirm the identity of samples using
vertebral counts or molecular methods, Harry \emph{et al.} (2012) found
only a single \emph{C. tilstoni} among a sample of 100 individuals from
the same location. All unidentified individuals were assumed to be
\emph{C. limbatus}.

\hypertarget{age-determination-and-growth-analysis}{%
\subsubsection{Age determination and growth
analysis}\label{age-determination-and-growth-analysis}}

Age was determined by counting concentric pairs of hypo- and
hyper-mineralised growth zones on thin-sections of vertebrae centra.
Ageing was conducted simultaneously with three other large carcharhinid
species also captured in the OTLF and a full description of ageing
protocols is given in Geraghty \emph{et al.} (2013). Ageing data for
\emph{Carcharhinus limbatus} were withheld from that publication due to
uncertainty about the effects of hybridisation with \emph{C. tilstoni}
(Morgan \emph{et al.} 2012).

Unstained, sectioned centra were aged twice under a microscope using
transmitted light by two readers independently without knowledge of
size, sex, or date of capture. \emph{Carcharhinus limbatus} has a
synchronous reproductive cycle in the region (see below), enabling
partial ages to be assigned subject to the assumptions that all growth
zones represented annual increments and the population birth date was
1st November (see Results). Within and between reader ageing bias was
investigated graphically using age-bias plots and statistically using
Bowker's test of symmetry (Hoenig \emph{et al.} 1995). Precision and
reproducibility in ageing was determined by calculating the coefficient
of variation of ageing error (Ogle 2017).

Growth was modelled using a sex-structured von Bertalanffy model
extended to explicitly include ageing error as a random effect, describe
variability in growth and length observation error, and incorporate
additional information on length at birth. The model described by Cope
and Punt (2007) was used as the basis for this approach. This assumes
that observed length at age, \(l_{i,g}\) for individual \emph{i} of sex
\(g\) can be described as:
\[l_{i,g} = L_{i,g} + \epsilon \quad\quad \epsilon \sim N(0,\sigma_{L,i,g}^2)\]
where \(L_{i,g}\) is the sex-specific expected length at age, calculated
using a three parameter von Bertalanffy function, modified to
incorporate length at birth:
\[L_{i,g}(a) = L_0 + (L_{\infty,g}-L_0)(1-e^{-K_{g}\cdot a_{i,g}})\]
where \(L_\infty\) is asymptotic length, \(L_0\) is length at birth,
\(K\) is the growth coefficient, and \(a_{i,g}\) is true age. Finally
\(\sigma_{L,i,g}\) is the standard deviation of the normally distributed
process error. \(L_\infty\) and \(K\) were assumed to be different for
males and females, given that size differences between sexes are
well-documented for this genus, while \(L_0\) and \(\sigma_{L,i,g}\)
were assumed to be the same for both sexes. As per Cope and Punt (2007),
the coefficient of variation of the process error, \(CV_L\) was assumed
to be proportional to length, such that for any individual
\(\sigma_{L,i,g} = CV_L L_{i,g}\). That is, there is more individual
variability in length at age for larger sharks.

Ageing error caused by the inconsistent interpretation of growth zones
by different readers was explicitly incorporated in the model as a
random effect by assuming that observed ages and true ages were linked
through the following relationship:
\[A_{i,j} = a_i + \epsilon_{a,ij} \quad \quad \epsilon_{a,ij} \sim N(0,\sigma_{a,i}^2)\]
where \(A_{i,j}\) is the observed age of individual \emph{i} by reader
\emph{j}, \(a_i\) is true age, and \(\sigma_{a,i}\) is the standard
deviation of ageing error. The coefficient of variation of ageing error
was assumed to be constant, such that, for any individual
\(\sigma_{a,i} = CV_{a}a_i\), with \(CV_a\) calculated outside the model
and approximated by the formula:
\[\hat{CV_a} = \sqrt{\frac{\sum_{i=1}^{n}CV_{A,i}^2}{n}}\] where
\(CV_{A,i}\) is the age coefficient of variation for individual \emph{i}
(Cope and Punt 2007).

Lastly, additional information on the \(L_0\) parameter was available
from neonates sampled in Moreton Bay that were known to have a true age
of zero, based on the presence of an unhealed umbilical scar. These
individuals were included in the statistical model through the
relationship
\[L_{0,i} = L_0 + \epsilon \quad \quad \epsilon \sim N(0,\sigma_{L,i}^2)\]
where \(L_{0,i}\) were the lengths of age zero individuals, and where
\(CV_L\) was again assumed proportional to length. The final model
contained a total of six parameters, and estimation was undertaken using
maximum likelihood with Template Model Builder (Kristensen \emph{et al.}
2016).

Validation of growth using traditional techniques (e.g.~fluorochrome
chemical marking, centrum edge methods) was not possible due to small
sample sizes and lack of suitable sampling across different months
within years. Indirect validation of early growth was undertaken by
comparing the fitted vertebral growth curve against monthly length
frequency of neonate and young juvenile sharks in Moreton Bay.

\hypertarget{reproductive-biology}{%
\subsubsection{Reproductive biology}\label{reproductive-biology}}

The relationship between weight, \emph{W}, and total length, \emph{L},
was modelled using a power curve
\(W(L) = \beta_1\cdot L^{\beta_2}e^{\epsilon}\). Parameters \(\beta_1\),
\(\beta_2\) and \(\sigma\) were obtained using maximum likelihood
estimation from the log-linear regression model
\(ln(W_i) = ln(\beta_1) + \beta_2\cdot ln(L_i) + \epsilon\), where
\(ln(\beta_1)\) and \(\beta_2\) correspond to the intercept and slope,
and where \(\epsilon \sim N(0, \sigma^2)\) is a normally distributed
random variable with variance \(\sigma^2\). Sex and the interaction of
sex and length were included as factors in the model to test whether
weight-at-length differed between males and females.

Maturity status was determined macroscopically using a three-stage
classification of male clasper condition (C = 1-3) and a six-stage
classification of female uterus condition (U = 1-6) (Table 1) (Walker
2007). Maturity stage data were then converted to binary form (immature
= 0, mature = 1) for statistical analysis. The relationship between
maturity stage and length was determined using a Generalised Linear
Model of the form
\[ P_i \sim Bernoulli(p_i) \quad\quad p_i = logit(\beta_3 + \beta_4 L_i)^{-1}\]
where the probability of an individual being mature, \(P_i\), was
modelled as a Bernoulli random variable, and parameters \(\beta_3\) and
\(\beta_4\) were estimated using maximum likelihood. The lengths at
which 50\% and 95\% of the population was mature were further derived
as:
\[L_{50} = -\beta_3/\beta_4 \quad\quad L_{95}=\frac{1}{\beta_4}log(19)-\beta_3/\beta_4\]
Bootstrapping (\emph{n} = 1000) was done on the derived parameters
(\(L_{50}\) and \(L_{95}\)) to estimate standard errors and calculate
confidence intervals. Maturity analyses were undertaken for both sexes
separately and combined, and also undertaken for age by substituting
with length in the above equation. The male maturation process was
further investigated by modelling the development of clasper length as a
function of length (see Supplementary Material).

For each analysis a graphical comparison was done for \emph{C. tilstoni}
using parameters estimated from previous studies (Stevens and Wiley
1986; Davenport and Stevens 1988; Harry \emph{et al.} 2013), and using
raw data presented in Harry \emph{et al.} (2013) for the Qld population
(see Fig. 1 for location of previous studies).

\hypertarget{demographic-analysis}{%
\subsubsection{Demographic analysis}\label{demographic-analysis}}

Comparative population dynamics was investigated using an age- and
sex-structured demographic model based on Lotka's generalized equation
(Xiao and Walker 2000; Xiao 2002):
\[1 = \int_{a_0}^{\infty}R(a)\beta(a)e^{\int_{a_0}^{a}\Lambda_{g, N}(s) ds}da\]
where \(a\) is age (years) and \(a_0\) = 0 is the age at birth,
\(R(a) = R\) is age-specific embryonic sex-ratio (assumed to be
constant), \(\beta(a)\) is age-specific female reproductive rate, and
\(\Lambda_{g, N}\) is the intrinsic rate of population decrease with age
of sex \emph{g} in numbers (year\(^{-1}\)). The age-specific female
reproductive rate was further given by:
\[\beta(a) = (\beta_5 + \beta_6 \cdot L_f(a)) \times P_{Max} \times (1 - exp( -ln(19)(\frac{a-A_{50}}{A_{95} - A_{95}})))^{-1}\]
where \(\beta_5\) and \(\beta_6\) are the intercept and slope of a
linear regression of fecundity at length and where \(P_{Max}\) is the
proportion of females in the population that reproduce annually.

\(\Lambda_{g,N}\) can be solved numerically if \(R(a)\) and \(\beta(a)\)
are known, and is a measure of the gross productivity of the population
(Xiao and Walker 2000). If female natural mortality at age, \(M_f(a)\),
is known and under the assumption of a stable age distribution, it is
possible to calculate the intrinsic rate of population increase with
time, \(r\), or the net productivity of the population for both sexes
and the gross productivity for males (Xiao and Walker 2000; Xiao 2002):
\[\Lambda_{f,N}(a) = -(r + M_f(a))\]
\[\Lambda_{m,N}(a) = -(r + M_m(a))\]

The stable age distribution of the population, the proportion of
individuals of sex \(g\), at age \emph{a}, in biomass was calculated as:
\[p_{g, B}(a) = \frac{W_{g}(a)exp(\int_{a_0}^{a}\Lambda_{g, N}(s) ds)}{\int_{a_0}^{\infty}W_{g}(a)exp(\int_{a_0}^{a}\Lambda_{g, N}(s) ds)da}\]
The mean and variance of ages in the population, in biomass, were
further calculated as: \[\mu = \int_{a_0}^{a}(a-a_0)p(a)da\] and
\[\sigma^2 = \int_{a_0}^{a}(a-a_0-\mu)p(a)da\]

The per-generation rate of multiplication, \(R_0\), or number of
daughters expected per female per lifetime, was calculated as:
\[R_0 =\int_{a_0}^{\infty}R(a)\beta(a)e^{\int_{a_0}^{a}M_f(s) ds}da \]

All numerical integration and optimization required in the above
equations was done using R (R Core Team 2018).

The influence of uncertainty in demographic traits on net productivity
(\(r\)) was investigated with Monte Carlo simulation (Cortés 2002;
Braccini \emph{et al.} 2015). For each species the demographic analysis
was repeated 1000 times using, where possible, weight, growth, and
maturity parameters randomly drawn from a multivariate normal
distribution. The mean and variance-covariance matrix of these
distributions were obtained from each of the respective best fit models
described above. Uncertainty in reproductive frequency was incorporated
by drawing random values of \(P_{Max}\) from a uniform distribution. A
constant, size-based method was used to specify natural mortality
\(M_g(a)\) in the above equations (Then \emph{et al.} 2015):
\[M_g = 4.118 \cdot K_g^{0.73} \cdot L_{\infty,g}^{-0.33}\] Additional
variability was added to the calculated value of \(M_g\) in each
simulation to ensure a wide range of values were considered. Values were
drawn from a random normal distribution with a CV of \(0.2M_g\).
Additional details of the Monte Carlo simulation and choice of priors is
supplied in the Supplementary Material.

Finally it was necessary to clarify two issues regarding interpretation
of the above demographic method, namely 1) the rationale behind the
choice of \(M\), and 2) what the assumed population depletion was. As
with most population models applied to fish, \(M\) was central to
interpretation but unknown. Many methods exist for calculating \emph{M}
ranging from purely theoretical to empirical (Kenchington 2014; Then
\emph{et al.} 2015), however the choice of method is subjective. A
common approach to dealing with this subjectivity is to average \emph{M}
over multiple methods, although Then \emph{et al.} (2015) advise against
this. A further problem, as noted by Pardo \emph{et al.} (2016), is that
many of these methods are based on data for teleosts, and do not give
reasonable values for sharks and rays. The second, interrelated issue is
the amount of population depletion corresponding to the life history
data used in the model (Cortés 2016). Since life history parameters
change in response to population density (Taylor and Gallucci 2009;
Romine \emph{et al.} 2013), unless all components of the data were
collected simultaneously and at a population size close to zero, the
final value of \(r\) is unlikely to reflect the maximum rate achievable,
\(r_m\). It is arguably \(r_m\) that is of greatest interest to fishery
managers (Pardo \emph{et al.} 2016).

One benefit of the formulation of the Euler-Lotka equation used in this
study is the explicit link between \(r\), \(\Lambda_f\), and \(M_f\)
(Xiao and Walker 2000). For a population to be viable
(i.e.~\(r \geq 0\)), \(M_f\) is restricted such that
\(0 \leq M_f \leq -\Lambda_f\). Assuming that density dependent changes
in \(r\) are predominantly mediated by changes in \emph{M} (Smith
\emph{et al.} 1998), then gross productivity of females, \(\Lambda_f\),
which is solely a function of reproductive characteristics and
independent of \emph{M}, effectively gives an upper bound for both \(r\)
and \(M\). The decision to use the method by Then \emph{et al.} (2015)
for calculating \emph{M} was somewhat arbitrary, however it was one of
the few methods that consistently led to plausible values for all
populations. The values of \emph{M} generated with this method were
slightly higher than the method used by Pardo \emph{et al.} (2016) in
their comparative study of \(r_m\) in sharks and rays. Nonetheless,
since most values corresponded to \textgreater{} 90\% annualised
survival (see below), it would be reasonable to interpret them as close
to the \(r_m\) that would occur at low population density.

\hypertarget{results}{%
\subsection{Results}\label{results}}

A total of 475 samples were available for life history analysis
including 183 females (68 - 267 cm) and 292 males (65 - 255 cm). Sharks
from Moreton Bay were predominantly neonates and small juveniles up to
127 cm and were captured in an approximately equal sex ratio (Fig. 2).
Sharks caught in northern NSW by the OTLF were predominantly
\textgreater{} 150 cm TL, and there was a bias towards males, which made
up 76\% of samples. Ninety percent of NSW samples were obtained from
depths between 30 and 90 m, with males caught to a maximum depth of 101
m, and females to 128 m (Fig. S1).

Of the 109 individuals sampled from NSW between 2008 and 2010 and
characterised using mtDNA and nDNA genotyping, 54 were identified as
purebred \emph{C. limbatus}. Sixteen individuals were hybrids that
displayed a conflict between their mtDNA and nDNA. A further 36
individuals were identified as possible hybrids carrying one or more
nuclear alleles from \emph{C. tilstoni}. Finally, two individuals were
unable to be identified and a single individual was identified as a
purebred \emph{C. tilstoni}. Like previous studies, all individuals with
hybrid ancestry appeared to have a \emph{C. limbatus} phenotype (Harry
\emph{et al.} 2012; Morgan \emph{et al.} 2012), consistent with a recent
study indicating that hybrids are introgressed (J.A.T. Morgan,
unpublished data). Hybrid individuals were included with the analysis of
\emph{C. limbatus} life history and the single purebred \emph{C.
tilstoni} was removed.

\hypertarget{age-and-growth}{%
\subsubsection{Age and growth}\label{age-and-growth}}

Distinct pairs of growth zones were visible in the corpus calcareum and,
to a lesser extent, the intermedialia of sectioned vertebrae (Fig. S2).
An age-bias plot showed discrepency between readers (Fig. S3), however
it was not statistically significant (Bowkers test of symmetry:
\(\chi^2\) = 35.33, d.f. = 34, \emph{P} = 0.405). Mean inter-reader CV
was 5.2 \%. On this basis an acceptable and reproducible interpretation
was considered to have been achieved, and further analyses proceeded
under the assumption that each pair of growth zones corresponded to an
annual increment.

Vertebrae sections were available from a total of 102 individuals.
Females (n = 33, 73 - 267 cm) ranged in age from 2 months to 22 years.
Males (n = 69, 74 - 252 cm) ranged in age from 2 months to 24 years. A
further 136 neonates collected in Moreton Bay were used for estimation
of the \(L_0\) parameter.

Growth of male and female \emph{C. limbatus} was similar until around
three years of age (Figs 3a-b and 3e) after which females had a greater
average length at age than males. Although absolute female growth rate
exceeded that of males, females attained a larger asymptotic length
(Table 2), and consequently \emph{K} was greatest for males. While males
attained a greater maximum age than females, this possibly reflects the
smaller sample sizes for females. The use of a CV proportional to length
resulted in a good fit to the data, as evident from the prediction
intervals for length at age data (Figs. 3 a \& b) and empirical \(L_0\)
observations (Fig. 3c). Variability in length, \(CV_L\), averaged 4.9 \%
of length (Table 2, Fig. 3d).

Monthly length measurements from a \emph{C. limbatus} nursery area in
Moreton Bay appeared to show distinct groupings corresponding to cohorts
from 0+ to 2+ years, most evident in October and November. Neonates with
open umbilical scars first appeared in Moreton Bay during early
November, at which time they were clearly separable from the 0+ and 1+
cohorts (Fig. 3f). A further seven individuals that were larger again
were thought to belong to the 2+ cohort. In comparison to the growth
curve fit to vertebral ages, sharks within Moreton Bay appeared to grow
slower (Fig. 3f).

\hypertarget{reproductive-biology-1}{%
\subsubsection{Reproductive biology}\label{reproductive-biology-1}}

Weight measurements were obtained from 100 females and 168 males (Fig.
4a). The heaviest female and male weighed 67 kg and 116 kg,
respectively. Although the heaviest individual was a male, no large
(\textgreater{} 240 cm) or pregnant females were weighed, and it is
likely they would exceed the weight of males of equivalent length.
Neither sex nor the interaction of sex and length were statistically
significant (ANOVA: d.f. = 2, 266, F = 1.444, \emph{P} = 0.2378) (Fig.
4b, Table 2).

Maturity stage was recorded for 171 females; the majority (n = 148) of
these were immature, most of which were neonates sampled from Moreton
Bay. The largest and oldest immature females (U = 1-2) were 208 cm and 8
years, respectively, while the smallest and youngest mature females (U =
3-6) were 199 cm and 7 years, respectively. Maturity stage was recorded
for 287 males. The largest and oldest immature males (C = 1-2) were 213
cm and 8 years, respectively, while the smallest and youngest mature
males (C = 3) were 190 cm and 8 years, respectively. For both age and
length, neither sex, nor the interaction with sex was statistically
significant (Analysis of deviance - length: \(\chi\) = 0.4095, d.f. = 2,
\emph{P} = 0.8149; Analysis of deviance - age: \(\chi\) = 1.652, d.f. =
2, \emph{P} = 0.4377, Fig. 5a-b, Table 2). Due to the small sample size
of mature females, it was not possible to model maternal condition and
assess the length and age at which reproduction begins. Male clasper
development was minimal until around 150 cm. Beyond this length claspers
began to elongate, reaching \(CL_{50}\) at 194 cm, around the same
length as maturity occurred (Fig. S4). Unlike \emph{C. tilstoni},
clasper length did not reach a well-defined asymptotic length.

Only 5 of the 23 mature females (U = 3-6), were pregnant (Table 3), so
limited inferences could be made about the timing and frequency of the
reproductive cycle. Pregnant females included three individuals at a
similar stage of embryonic development with an overall mean embryo
length of 44.6 cm during April, and another in June with a mean embryo
length of 44.4 cm. Mean fecundity of females was 6.6 pups, and ranged
from 2 to 9. The mean (\(\pm\) s.d.) length of 136 neonates from Moreton
Bay was 73.09 \(\pm\) 3.43 cm. The mean (\(\pm\) s.d.) weight of 103
neonates was 2.156 \(\pm\) 0.408 kg.

In the four years that samples were collected from Moreton Bay, neonates
were present from early November to early February, suggesting \emph{C.
limbatus} has a synchronous reproductive cycle with a three-month
pupping season from November to January. Four post-partum (U = 6)
females were also recorded between April and June. This conflicted with
other evidence suggestive of a synchronous reproductive cycle and it was
thought these females might have aborted their young upon capture or
been assigned an incorrect uterus stage. The fraction of mature females
that were pregnant was 22\%, or 39\% if U = 6 females were included.
Although it was not possible to determine the duration and frequency of
reproduction, the large proportion of non-pregnant (U = 3) females
observed suggests that it is likely to be a minimum of two years in
duration or even longer.

\hypertarget{demographic-analysis-1}{%
\subsubsection{Demographic analysis}\label{demographic-analysis-1}}

Gross population productivity of females, \(\Lambda_f\), was highest for
\emph{C. tilstoni}, with mean values for the NT 19 \% higher than that
of Qld and 63\% higher than \emph{C. limbatus} (Table 4). Holding all
other variables equal, the range of possible values for \(M_f\) are thus
\(0 \leq M_f \leq -\Lambda_f\), for a population to be viable
(\(r \geq 0\)). Values for \(M_f\) predicted from the size-based method
were within the lower half of plausible values (Fig. 6b); mean
annualised survival (\(e^{-M_f}\)) ranged from 93 \% in \emph{C.
limbatus} to 90\% in NT \emph{C. tilstoni}. Due to the high CV
specified, a wide range of values were nonetheless included in the Monte
Carlo simulation (Fig. 6a). In line with its higher \(M\), net
productivity, \(r\), was higher for \emph{C. tilstoni} (Fig. 6b). The
net productivity of the NT population was 18 \% higher than Qld and 82\%
higher than \emph{C. limbatus}. Reproductive output, \(R_0\), was
variable across all species, and highest on average for Qld \emph{C.
tilstoni} with females producing, on average, 12 female pups during
their life.

Due to their smaller \(L_\infty\) and higher \(K\) (Table 2), average
\(M\) calculated using the size-based method was higher for males than
females (Table 4). This translated to annualised male survival being 1\%
lower in \emph{C. limbatus} and 3\% lower in \emph{C. tilstoni}.
Accordingly, the mean age of the population in numbers and biomass was
always greater in females than males (Fig. 6c, Table 4). Mean age in
both \emph{C. tilstoni} populations was 3 to 4 years, or 5 to 7 years if
weighted by biomass. Mean age in the \emph{C. limbatus} population was
c. 5 years, or c. 10 years if weighted by biomass. As a result of this
difference, biomass in \emph{C. limbatus} populations is predominantly
concentrated in lengths greater than the maximum size of \emph{C.
tilstoni} (Fig. 6d).

\hypertarget{discussion}{%
\subsection{Discussion}\label{discussion}}

\hypertarget{life-history-of-c.-limbatus}{%
\subsubsection{\texorpdfstring{Life history of \emph{C.
limbatus}}{Life history of C. limbatus}}\label{life-history-of-c.-limbatus}}

This study is the first detailed description of the life history of
\emph{C. limbatus} from Australian waters, where information has
previously been difficult to obtain due to its co-occurrence with the
morphologically similar \emph{C. tilstoni}. Aspects of the life history
of \emph{C. limbatus} have previously been studied in many parts of its
range including the northwest Atlantic (Castro 1996), South Africa
(Dudley and Cliff 1993; Wintner and Cliff 1996), Indonesia (White 2007;
Smart \emph{et al.} 2015), Senegal (Capape \emph{et al.} 2004), and,
most comprehensively, the Gulf of Mexico (Branstetter 1987; Killam and
Parsons 1989; Carlson \emph{et al.} 2006; Tovar-Ávila \emph{et al.}
2009; Passerotti and Baremore 2012; Baremore and Passerotti 2013).
Generally speaking, Australian \emph{C. limbatus} has life history
characteristics least similar to populations in the Gulf of Mexico and
northwest Atlantic (GOM and NWA), which are characterised by smaller
maximum lengths that rarely exceed 200 cm. Empirical maximum lengths
from the present study demonstrate that in Australian waters both sexes
regularly exceed 250 cm. This is somewhat larger still than populations
off South Africa, Indonesia, and Senegal where maximum lengths of 245 to
247 cm were recorded (Dudley and Cliff 1993; Capape \emph{et al.} 2004;
White 2007). The heaviest male from this study (241cm) weighed 116 kg
and the largest female (267cm) was predicted to weigh between 89 and 153
kg (Fig. 4a). A 221 cm, 122 kg mature non-pregnant (U = 3) female
captured in a Qld recreational fishing tournament appears to be the
heaviest measured specimen both in Australia and globally (Salini
\emph{et al.} 2007).

Conclusive validation of growth has not yet been achieved for any
\emph{C. limbatus} population. Growth estimates from independent
vertebral ageing studies in the Gulf of Mexico have consistently been
similar (Carlson \emph{et al.} 2006; Tovar-Ávila \emph{et al.} 2009;
Passerotti and Baremore 2012), providing some confidence that vertebral
growth zones in this species reflect age. No validation was possible in
the current study although a visual comparison of the monthly length
structure from Moreton Bay suggested a slightly slower growth than was
estimated from vertebral sections. Lengths at age of \emph{C. limbatus}
off central eastern Australia were intermediate to South African and
Indonesian populations (Wintner and Cliff 1996; Smart \emph{et al.}
2015). However, noting that each of these studies are unvalidated and
based on small sample sizes, it is probably unwise to read any
biological significance into differences among those populations.
Incorrect interpretation of the first growth increment or a single early
growth zone could explain the difference between studies. The oldest
individuals in this study were a 24 year old male and a 22 year old
female. These maximum ages were similar to the oldest \emph{C. limbatus}
aged from the GOM (Passerotti and Baremore 2012), a 23.5 year old male.
Noting the potential for age underestimation in shark and ray ageing
studies (Harry 2018), it is possible that \emph{C. limbatus} in all
populations live longer than currently recognised.

The growth modeling approach in the current study involved adapting the
model of Cope and Punt (2007). A key feature of their model is the
explicit incorporation of ageing error as a random effect. This is a
useful addition for sharks and rays, which can be difficult to age
(Goldman \emph{et al.} 2012). In this study the model was changed to
incorporate two relevant features of \emph{C. limbatus} biology.
Firstly, it was fit to both sexes simultaneously. This avoided
unnecessarily estimating sex-specific parameters for \(L_0\), which was
the same for both sexes. It also avoided estimating a sex-specific
variance parameter, which was assumed not to differ between sexes
either. Secondly, empirical data on \(L_0\) were included in the
statistical model itself, and used to estimate \(L_0\) with vertebral
ageing data jointly. This approach removes the potential bias in
parameter estimates caused by the more common approach of fixing \(L_0\)
at a known value (Pardo \emph{et al.} 2013).

With few pregnant females sampled and no data collected on ovarian
follicles, many aspects of reproduction remain uncertain for central
eastern Australian \emph{C. limbatus}. The seasonal occurrence of large
numbers of neonates in Moreton Bay between November and February (Taylor
and Bennett 2013) indicates that reproduction is synchronous and occurs
in summer, as is typical throughout its range (Fourmanoir 1961;
Simpfendorfer and Milward 1993; Castro 1993). Less is known about the
frequency of reproduction. Within both the GOM and NWA, \emph{C.
limbatus} reproduces biennially (Castro 1996; Baremore and Passerotti
2013). Capape \emph{et al.} (2004) also speculated that \emph{C.
limbatus} had a biennial reproductive cycle off Senegal. Off South
Africa, patterns in development of ovarian follicles sampled from 260
mature females supported a three year reproductive cycle (Dudley and
Cliff 1993). The high proportion of mature, non-pregnant females in this
study area indicates that reproduction is at least biennial and likely
longer. This would be consistent with other large congeners such as
dusky, \emph{Carcharhinus obscurus} (Dudley \emph{et al.} 2005), and
sandbar, \emph{Carcharhinus plumbeus}, sharks (Baremore and Hale 2012).
Alternatively, sampling limitations including spatial, temporal and gear
selectivity effects might have contributed to the low proportion of
pregnant females observed.

Despite limited data, the average fecundity of 6.6 calculated in this
study is consistent with other populations that attain a similar maximum
size. Mean and median litter sizes of 6.7 and 6, respectively, were
reported in two studies off South Africa, where litter sizes ranged from
1 to 11 (Bass \emph{et al.} 1973; Dudley and Cliff 1993). In Indonesia,
mean litter size was 6.6 and ranged from 2 to 10 (White 2007), while off
Senegal, mean litter size was 6.8 and ranged from 1 to 8 (Cadenat and
Blache 1981; Capape \emph{et al.} 2004). Variability in length at birth
is well documented in \emph{C. limbatus} (Garrick 1982; Harry \emph{et
al.} 2012). Australian \emph{C. limbatus} are born at a slightly longer
length to other global populations, although length and weight at birth
was similar to the single full-term litter measured off South Africa
(Dudley and Cliff 1993). Harry \emph{et al.} (2012) speculated that this
larger length could be a way of reducing competition with \emph{C.
tilstoni}, which co-occurs in some nursery areas within Australia
(Simpfendorfer and Milward 1993).

Length at maturity in Australian waters appears to be most similar to
South Africa, where both sexes mature at \textgreater{} 200 cm, although
robust comparison between studies is not possible due to methodological
differences (Dudley and Cliff 1993). While male and female \emph{C.
limbatus} in Australian waters probably differ in their length and age
at maturity, data were insufficient to demonstrate this using
statistical hypothesis testing, nor were there sufficient data to
determine when females begin to reproduce. Baremore and Passerotti
(2013), who examined length at maternity of \emph{C. limbatus} in the
GOM, found that 50\% of females were in maternal condition at 10.1
years. Based on the age at maturity of c. 8 years and a two to three
year reproductive cycle, it is unlikely that females would start
contributing to population recruitment before age 10 off central eastern
Australia.

Additional notes on the ecology of \emph{C. limbatus} are provided in
the Supplementary Material.

\hypertarget{comparative-demography-of-c.-limbatus-and-c.-tilstoni}{%
\subsubsection{\texorpdfstring{Comparative demography of \emph{C.
limbatus} and \emph{C.
tilstoni}}{Comparative demography of C. limbatus and C. tilstoni}}\label{comparative-demography-of-c.-limbatus-and-c.-tilstoni}}

This study confirms that the co-occurring \emph{C. limbatus} and
\emph{C. tilstoni} differ substantially in many aspects of their life
histories, despite their morphological similarities. At birth, \emph{C.
limbatus} are approximately the same size as one year old \emph{C.
tilstoni} (Figs 3 a \& b). Length at age between species further
diverges over ontogeny with \emph{C. limbatus} remaining larger, on
average, throughout life (Fig. 3f). Although weight at length is similar
between species (Fig. 4b), there are large differences in maximum body
size, with the heaviest \emph{C. limbatus} more than three times the
weight of the heaviest \emph{C. tilstoni}. Maturation of \emph{C.
limbatus} begins at lengths well above the maximum length attained by
\emph{C. tilstoni} (Fig. 5c), which also matures and reproduces at a
younger age (Fig. 5d). The larger \emph{C. limbatus} has a higher mean
fecundity (Table 2), but likely reproduces less frequently. Based on a
long-term tagging study, \emph{C. tilstoni} is known to live to at least
20 years (Last and Stevens 2009; Harry \emph{et al.} 2013). Vertebral
ageing from this study suggested \emph{C. limbatus} live to 24 years.
Given their substantially larger size and likely lower mortality, as
well as the potential for age underestimation (Harry 2018), it is
reasonable to surmise that \emph{C. limbatus} lives longer than \emph{C.
tilstoni}.

Differences in life history between the Qld and NT \emph{C. tilstoni}
populations are summarised in Harry \emph{et al.} (2013). While many
aspects of their biology are similar, Harry \emph{et al.} (2013) found
that the Qld population matured at an older age and larger size. Growth
of the NT population was faster than Qld and, although not reflected in
growth parameters, the Qld population attains a slightly larger length.
Since fecundity is proportional to length and length at birth is similar
between populations, the lower average fecundity of the NT population is
likely a result of mature females being smaller, on average, than those
from Qld.

The values of \(r\) calculated in this study suggest that both \emph{C.
limbatus} and \emph{C. tilstoni} are moderately productive species
relative to other sharks and rays (Simpfendorfer and Dulvy 2017).
Relative to each other however, \emph{C. tilstoni} has a much higher
productivity than \emph{C. limbatus}, meaning these species are likely
to differ in their response to exploitation. Further confounding these
differences are the divergent size structures of the two species. When
biomass is plotted against length, the fundamental differences between
the life histories of the two species become particularly apparent (Fig.
6d). Whereas the smaller \emph{C. tilstoni} completes its entire life
cycle within nearshore and coastal habitats, this habitat is used only
transitionally by \emph{C. limbatus} as juveniles, with the majority of
the population biomass occurring further offshore.

These differences in demography have important implications for the
numerous fisheries that capture this species complex within Australian
waters and should be considered in the formulation of appropriate
management strategies. For size-selective gillnet fisheries capturing
these species, it is evident that the majority of the \emph{C. limbatus}
biomass will be less vulnerable to fishing. Viewed in the historical
context, large-scale, historical exploitation by the Taiwanese fleet
would probably have had minimal impact on the adult biomass of \emph{C.
limbatus}, perhaps explaining the species' increase in relative
abundance in the subsequent decades (Ovenden \emph{et al.} 2010).
Likewise, the impacts of management measures such as size limits will
affect each species differently. For example, in Qld where sharks are
often captured by recreational fishers (Lynch \emph{et al.} 2010), the
upper size limit of 1.5m for this sector would presumably be ineffective
at reducing mortality on \emph{C. tilstoni} and effective for \emph{C.
limbatus}. Thus, despite the latter being inherently more susceptible to
exploitation, \emph{C. tilstoni} will likely have a greater exposure to
inshore mixed-species fisheries.

At present, all Australian jurisdictions report \emph{C. tilstoni} and
\emph{C. limbatus} together. Morphological similarities aside, there is
little, if any, justification for such grouping. Given the likely
divergent responses to exploitation of the two species, it is likely
that any inferences from catch or catch rates would be invalid, masking
underlying population trends. Currently, exploitation of blacktip shark
is low across northern Australia, however historically all regions and
stocks have sporadically experienced periods of high catch driven by
cyclical market trends. Any future attempts to develop these resources
should acknowledge the differences between these species and ensure
appropriate management measures are in place that include routine
monitoring of relative abundance.

\hypertarget{acknowledgements}{%
\subsection{Acknowledgements}\label{acknowledgements}}

The authors are indebted to the efforts of the Qld and NSW commercial
shark fishers who enabled this research, as well as the scientific
observers and laboratory staff from the NSW Department of Primary
Industries who assisted in sample collection and analysis. Thank you to
DS Waltrick and LFN Waltrick for your support and encouragement. This
research was carried out using pre-existing data from completed research
projects and as such did not receive any specific funding.

\hypertarget{conflicts-of-interest}{%
\subsection{Conflicts of Interest}\label{conflicts-of-interest}}

The authors declare no conflicts of interest.

\newpage

\hypertarget{references}{%
\subsection*{References}\label{references}}
\addcontentsline{toc}{subsection}{References}

\hypertarget{refs}{}
\leavevmode\hypertarget{ref-baremore_reproduction_2012}{}%
Baremore IE, Hale LF (2012) Reproduction of the Sandbar Shark in the
Western North Atlantic Ocean and Gulf of Mexico. \emph{Marine and
Coastal Fisheries} \textbf{4}, 560--572.
doi:\href{https://doi.org/10.1080/19425120.2012.700904}{10.1080/19425120.2012.700904}.

\leavevmode\hypertarget{ref-baremore_reproduction_2013}{}%
Baremore IE, Passerotti MS (2013) Reproduction of the Blacktip Shark in
the Gulf of Mexico. \emph{Marine and Coastal Fisheries} \textbf{5},
127--138.
doi:\href{https://doi.org/10.1080/19425120.2012.758204}{10.1080/19425120.2012.758204}.

\leavevmode\hypertarget{ref-bass_sharks_1973}{}%
Bass AJ, D'Aubrey JD, Kistnasamy N (1973) Sharks of the east coast of
southern Africa. 1. The genus \emph{Carcharhinus (Carcharhinidae)}.
Oceanographic Research Institute, 33. (Durban)

\leavevmode\hypertarget{ref-boomer_genetic_2010}{}%
Boomer JJ, Peddemors V, Stow AJ (2010) Genetic data show that
\emph{Carcharhinus tilstoni} is not confined to the tropics,
highlighting the importance of a multifaceted approach to species
identification. \emph{Journal of Fish Biology} \textbf{77}, 1165--1172.
doi:\href{https://doi.org/10.1111/j.1095-8649.2010.02770.x}{10.1111/j.1095-8649.2010.02770.x}.

\leavevmode\hypertarget{ref-braccini_displaying_2015}{}%
Braccini M, Brooks EN, Wise B, McAuley R (2015) Displaying uncertainty
in the biological reference points of sharks. \emph{Ocean Coastal
Management} \textbf{116}, 143--149.
doi:\href{https://doi.org/10.1016/j.ocecoaman.2015.07.014}{10.1016/j.ocecoaman.2015.07.014}.

\leavevmode\hypertarget{ref-bradshaw_more_2013}{}%
Bradshaw CJ, Field IC, McMahon CR, Johnson GJ, Meekan MG, Buckworth RC
(2013) More analytical bite in estimating targets for shark harvest.
\emph{Marine Ecology-Progress Series} \textbf{488}, 221--232.

\leavevmode\hypertarget{ref-brander_disappearance_1981}{}%
Brander K (1981) Disappearance of common skate \emph{Raia batis} from
Irish Sea. \emph{Nature} \textbf{290}, 48--49.
doi:\href{https://doi.org/10.1038/290048a0}{10.1038/290048a0}.

\leavevmode\hypertarget{ref-branstetter_age_1987}{}%
Branstetter S (1987) Age and Growth-Estimates for Blacktip,
\emph{Carcharhinus limbatus}, and Spinner, \emph{C. brevipinna,} Sharks
from the Northwestern Gulf of Mexico. \emph{Copeia} 964--974.

\leavevmode\hypertarget{ref-broadhurst_temporal_2014}{}%
Broadhurst MK, Butcher PA, Millar RB, Marshall JE, Peddemors VM (2014)
Temporal hooking variability among sharks on south-eastern Australian
demersal longlines and implications for their management. \emph{Global
Ecology and Conservation} \textbf{2}, 181--189.
doi:\href{https://doi.org/10.1016/j.gecco.2014.09.005}{10.1016/j.gecco.2014.09.005}.

\leavevmode\hypertarget{ref-cadenat_requins_1981}{}%
Cadenat J, Blache J (1981) `Requins de Méditerranée et l'Atlantique
(plus particulièrement de la côte occidentale d'Afrique).' (ORSTOM:
Paris)

\leavevmode\hypertarget{ref-capape_reproductive_2004}{}%
Capape C, Seck AA, Diatta Y, Reynaud C, Hemida F, Zaouali J (2004)
Reproductive biology of the blacktip shark, \emph{Carcharhinus limbatus}
(Chondrichthyes : Carcharhinidae) off west and north African coasts.
\emph{Cybium} \textbf{28}, 275--284.

\leavevmode\hypertarget{ref-carlson_differences_2006}{}%
Carlson JK, Sulikowski JR, Baremore IE (2006) Do differences in life
history exist for blacktip sharks, \emph{Carcharhinus limbatus}, from
the United States South Atlantic Bight and Eastern Gulf of Mexico?
\emph{Environmental Biology of Fishes} \textbf{77}, 279--292.

\leavevmode\hypertarget{ref-castro_shark_1993}{}%
Castro JI (1993) The shark nursery of Bulls Bay, South-Carolina, with a
review of the shark nurseries of the southeastern coast of the
United-States. \emph{Environmental Biology of Fishes} \textbf{38},
37--48.

\leavevmode\hypertarget{ref-castro_biology_1996}{}%
Castro JI (1996) Biology of the blacktip shark, \emph{Carcharhinus}
\emph{Limbatus}, off the southeastern United States. \emph{Bulletin of
Marine Science} \textbf{59}, 508--522.
doi:\href{https://doi.org/10.1007/s10641-006-9129-x}{10.1007/s10641-006-9129-x}.

\leavevmode\hypertarget{ref-compagno_sharks_1984}{}%
Compagno LJV (1984) Sharks of the world. An annotated and illustrated
catalogue of shark species known to date. FAO species catalogue,
Hexanchiformes to Lamniformes. FAO Fisheries Synopsis 125 Vol. 4, Part
1.

\leavevmode\hypertarget{ref-compagno_sharks_1988}{}%
Compagno LJV (1988) `Sharks of the order Carcharhiniformes.' (Princeton
University Press: New Jersey)

\leavevmode\hypertarget{ref-cope_admitting_2007}{}%
Cope JM, Punt AE (2007) Admitting ageing error when fitting growth
curves: An example using the von Bertalanffy growth function with random
effects. \emph{Canadian Journal of Fisheries and Aquatic Sciences}
\textbf{64}, 205--218.

\leavevmode\hypertarget{ref-cortes_incorporating_2002}{}%
Cortés E (2002) Incorporating uncertainty into demographic modeling:
Application to shark populations and their conservation.
\emph{Conservation Biology} \textbf{16}, 1048--1062.
doi:\href{https://doi.org/10.1046/j.1523-1739.2002.00423.x}{10.1046/j.1523-1739.2002.00423.x}.

\leavevmode\hypertarget{ref-cortes_perspectives_2016}{}%
Cortés E (2016) Perspectives on the intrinsic rate of population growth.
\emph{Methods in Ecology and Evolution} \textbf{7}, 1136--1145.
doi:\href{https://doi.org/10.1111/2041-210X.12592}{10.1111/2041-210X.12592}.

\leavevmode\hypertarget{ref-davenport_age_1988}{}%
Davenport S, Stevens JD (1988) Age and growth of two commercially
important sharks (\emph{Carcharhinus tilstoni} and \emph{C. sorrah})
from Northern Australia. \emph{Australian Journal of Marine and
Freshwater Research} \textbf{39}, 417--433.
doi:\href{https://doi.org/10.1071/MF9880417}{10.1071/MF9880417}.

\leavevmode\hypertarget{ref-dudley_sharks_1993}{}%
Dudley SFJ, Cliff G (1993) Sharks caught in the protective gill nets off
Natal, South-Africa 7. The blacktip shark \emph{Carcharhinus limbatus}
(Valenciennes). \emph{South African Journal of Marine Science}
\textbf{13}, 237--254.

\leavevmode\hypertarget{ref-dudley_sharks_2005}{}%
Dudley SFJ, Cliff G, Zungu MP, Smale MJ (2005) Sharks caught in the
protective gill nets off KwaZulu-Natal, South Africa. 10. The dusky
shark \emph{Carcharhinus obscurus} (Lesueur 1818). \emph{African Journal
of Marine Science} \textbf{27}, 107--127.

\leavevmode\hypertarget{ref-dulvy_fishery_2000}{}%
Dulvy NK, Metcalfe JD, Glanville J, Pawson MG, Reynolds JD (2000)
Fishery stability, local extinctions, and shifts in community structure
in skates. \emph{Conservation Biology} \textbf{14}, 283--293.
doi:\href{https://doi.org/10.1046/j.1523-1739.2000.98540.x}{10.1046/j.1523-1739.2000.98540.x}.

\leavevmode\hypertarget{ref-ebert_resurrection_2010}{}%
Ebert DA, White WT, Goldman KJ, Compagno LJV, Daly-Engel TS, Ward RD
(2010) Resurrection and redescription of \emph{Squalus suckleyi}
(Girard, 1854) from the North Pacific, with comments on the
\emph{Squalus acanthias} subgroup (Squaliformes: Squalidae).
\emph{Zootaxa} 22--40.

\leavevmode\hypertarget{ref-field_changes_2012}{}%
Field IC, Buckworth RC, Yang G-J, Meekan MG, Johnson G, Stevens JD,
Pillans RD, McMahon CR, Bradshaw CJ (2012) Changes in size distributions
of commercially exploited sharks over 25 years in northern Australia
using a Bayesian approach. \emph{Fisheries Research} \textbf{125},
262--271.
doi:\href{https://doi.org/10.1016/j.fishres.2012.03.005}{10.1016/j.fishres.2012.03.005}.

\leavevmode\hypertarget{ref-fourmanoir_requins_1961}{}%
Fourmanoir P (1961) Requins de la côte ouest de Madagascar.
\emph{Mémoires de l'Institut Scientifique de Madagascar Série F:
Océanographie} \textbf{4}, 3--81.

\leavevmode\hypertarget{ref-garrick_sharks_1982}{}%
Garrick JAF (1982) Sharks of the genus \emph{Carcharhinus} NOAA
Technical Report NMFS Circular 445.

\leavevmode\hypertarget{ref-geraghty_age_2013}{}%
Geraghty PT, Macbeth WG, Harry AV, Bell JE, Yerman MN, Williamson JE
(2013) Age and growth parameters for three heavily exploited shark
species off temperate eastern Australia. \emph{ICES Journal of Marine
Science: Journal du Conseil} fst164.
doi:\href{https://doi.org/10.1093/icesjms/fst164}{10.1093/icesjms/fst164}.

\leavevmode\hypertarget{ref-goldman_assessing_2012}{}%
Goldman KJ, Cailliet GM, Andrews AH, Natanson LJ (2012) Assessing the
age and growth of chondrichthyan fishes. `Biology of Sharks and Their
Relatives, Second Edition'. (Eds JC Carrier, JA Musick, MR Heithaus) pp.
423--451. (CRC Press: New York)

\leavevmode\hypertarget{ref-harry_evidence_2018}{}%
Harry AV (2018) Evidence for systemic age underestimation in shark and
ray ageing studies. \emph{Fish and Fisheries} \textbf{19}, 185--200.
doi:\href{https://doi.org/10.1111/faf.12243}{10.1111/faf.12243}.

\leavevmode\hypertarget{ref-harry_comparison_2012}{}%
Harry AV, Morgan JAT, Ovenden JR, Tobin AJ, Welch DJ, Simpfendorfer CA
(2012) Comparison of the reproductive ecology of two sympatric blacktip
sharks (\emph{Carcharhinus limbatus} and \emph{Carcharhinus tilstoni})
off north-eastern Australia with species identification inferred from
vertebral counts. \emph{Journal of Fish Biology}.
doi:\href{https://doi.org/10.1111/j.1095-8649.2012.03400.x}{10.1111/j.1095-8649.2012.03400.x}.

\leavevmode\hypertarget{ref-harry_age_2013}{}%
Harry AV, Tobin AJ, Simpfendorfer CA (2013) Age, growth and reproductive
biology of the spot-tail shark, \emph{Carcharhinus sorrah}, and the
Australian blacktip shark, \emph{Carcharhinus tilstoni}, from the Great
Barrier Reef World Heritage Area, north-eastern Australia. \emph{Marine
and Freshwater Research} \textbf{64}, 277--293.
doi:\href{https://doi.org/10.1071/MF12142}{10.1071/MF12142}.

\leavevmode\hypertarget{ref-harry_evaluating_2011}{}%
Harry AV, Tobin AJ, Simpfendorfer CA, Welch DJ, Mapleston A, White J,
Williams AJ, Stapley J (2011) Evaluating catch and mitigating risk in a
multi-species, tropical, inshore shark fishery within the Great Barrier
Reef World Heritage Area. \emph{Marine and Freshwater Research}
\textbf{62}, 710--721.
doi:\href{https://doi.org/10.1071/MF10155}{10.1071/MF10155}.

\leavevmode\hypertarget{ref-hatfield_ecological_1999}{}%
Hatfield T, Schluter D (1999) Ecological speciation in sticklebacks:
Environment-dependent hybrid fitness. \emph{Evolution} \textbf{53},
866--873.
doi:\href{https://doi.org/10.1111/j.1558-5646.1999.tb05380.x}{10.1111/j.1558-5646.1999.tb05380.x}.

\leavevmode\hypertarget{ref-hoenig_analyzing_1995}{}%
Hoenig JM, Morgan MJ, Brown CA (1995) Analyzing differences between two
age-determination methods by tests of symmetry. \emph{Canadian Journal
of Fisheries and Aquatic Sciences} \textbf{52}, 364--368.
doi:\href{https://doi.org/10.1139/f95-038}{10.1139/f95-038}.

\leavevmode\hypertarget{ref-iglesias_taxonomic_2010}{}%
Iglésias SP, Toulhoat L, Sellos DY (2010) Taxonomic confusion and market
mislabelling of threatened skates: Important consequences for their
conservation status. \emph{Aquatic Conservation: Marine and Freshwater
Ecosystems} \textbf{20}, 319--333.

\leavevmode\hypertarget{ref-johnson_novel_2017}{}%
Johnson GJ, Buckworth RC, Lee H, Morgan JAT, Ovenden JR, McMahon CR
(2017) A novel field method to distinguish between cryptic carcharhinid
sharks, Australian blacktip shark \emph{Carcharhinus tilstoni} and
common blacktip shark \emph{C. limbatus}, despite the presence of
hybrids. \emph{Journal of Fish Biology} \textbf{90}, 39--60.
doi:\href{https://doi.org/10.1111/jfb.13102}{10.1111/jfb.13102}.

\leavevmode\hypertarget{ref-kenchington_natural_2014}{}%
Kenchington TJ (2014) Natural mortality estimators for
information-limited fisheries. \emph{Fish and Fisheries} \textbf{15},
533--562.
doi:\href{https://doi.org/10.1111/faf.12027}{10.1111/faf.12027}.

\leavevmode\hypertarget{ref-killam_age_1989}{}%
Killam KA, Parsons GR (1989) Age and Growth of the Blacktip Shark,
\emph{Carcharhinus limbatus}, near Tampa Bay, Florida. \emph{Fishery
Bulletin} \textbf{87}, 845--857.

\leavevmode\hypertarget{ref-kristensen_tmb:_2016}{}%
Kristensen K, Nielsen A, Berg CW, Skaug H, Bell BM (2016) TMB: Automatic
Differentiation and Laplace Approximation. \emph{Journal of Statistical
Software} \textbf{70}, 1--21.
doi:\href{https://doi.org/10.18637/jss.v070.i05}{10.18637/jss.v070.i05}.

\leavevmode\hypertarget{ref-last_sharks_2009}{}%
Last PR, Stevens JD (2009) `Sharks and rays of Australia.' (CSIRO
Publishing: Collingwood)

\leavevmode\hypertarget{ref-lavery_genetic_1991}{}%
Lavery S, Shaklee JB (1991) Genetic evidence for separation of two
sharks, \emph{Carcharhinus limbatus} and \emph{C. tilstoni}, from
Northern Australia. \emph{Marine Biology} \textbf{108}, 1--4.
doi:\href{https://doi.org/10.1007/bf01313464}{10.1007/bf01313464}.

\leavevmode\hypertarget{ref-lyle_mercury_1984}{}%
Lyle JM (1984) Mercury concentrations in four carcharhinid and three
hammerhead sharks from coastal waters of the Northern Territory.
\emph{Marine and Freshwater Research} \textbf{35}, 441--451.
doi:\href{https://doi.org/10.1071/MF9840441}{10.1071/MF9840441}.

\leavevmode\hypertarget{ref-lynch_implications_2010}{}%
Lynch AJ, Sutton SG, Simpfendorfer CA (2010) Implications of
recreational fishing for elasmobranch conservation in the Great Barrier
Reef Marine Park. \emph{Aquatic Conservation: Marine and Freshwater
Ecosystems} \textbf{20}, 312--318.

\leavevmode\hypertarget{ref-macbeth_observer-based_2009}{}%
Macbeth WG, Geraghty PT, Peddemors VM, Gray CA (2009) Observer-based
study of targeted commercial fishing for large shark species in waters
off northern New South Wales. Cronulla Fisheries Research Centre of
Excellence, Industry \& Investment NSW, (Cronulla)

\leavevmode\hypertarget{ref-marshall_redescription_2009}{}%
Marshall AD, Compagno LJV, Bennett MB (2009) Redescription of the genus
\emph{Manta} with resurrection of \emph{Manta alfredi} (Krefft, 1868)
(Chondrichthyes; Myliobatoidei; Mobulidae). \emph{Zootaxa} 1--28.

\leavevmode\hypertarget{ref-millington_prospects_1981}{}%
Millington P, Walter D (1981) Prospects for Australian fishermen in
northern gillnet fishery. \emph{Australian Fisheries} \textbf{40}, 3--8.

\leavevmode\hypertarget{ref-morgan_detection_2012}{}%
Morgan JA, Harry AV, Welch DJ, Street R, White J, Geraghty PT, Macbeth
WG, Tobin A, Simpfendorfer CA, Ovenden JR (2012) Detection of
interspecies hybridisation in Chondrichthyes: Hybrids and hybrid
offspring between Australian (\emph{Carcharhinus tilstoni}) and common
(\emph{C. limbatus}) blacktip shark found in an Australian fishery.
\emph{Conservation Genetics} \textbf{13}, 455--463.
doi:\href{https://doi.org/10.1007/s10592-011-0298-6}{10.1007/s10592-011-0298-6}.

\leavevmode\hypertarget{ref-morgan_mitochondrial_2011}{}%
Morgan JAT, Welch DJ, Harry AV, Street R, Broderick D, Ovenden JR (2011)
A mitochondrial species identification assay for Australian blacktip
sharks (\emph{Carcharhinus tilstoni}, \emph{C. limbatus} and \emph{C.
amblyrhynchoides}) using real-time PCR and high-resolution melt
analysis. \emph{Molecular Ecology Resources} \textbf{11}, 813--819.
doi:\href{https://doi.org/10.1111/j.1755-0998.2011.03023.x}{10.1111/j.1755-0998.2011.03023.x}.

\leavevmode\hypertarget{ref-ogle_fsa:_2017}{}%
Ogle DH (2017) `FSA: Fisheries Stock Analysis. R package version 0.8.8.'

\leavevmode\hypertarget{ref-ovenden_towards_2010}{}%
Ovenden JR, Morgan JAT, Kashiwaga T, Broderick D, Salini J (2010)
Towards better management of Australia's shark fishery: Genetic analyses
reveal unexpected ratios of cryptic blacktip species \emph{Carcharhinus
tilstoni} and \emph{C. limbatus}. \emph{Marine and Freshwater Research}
\textbf{61}, 253--262.

\leavevmode\hypertarget{ref-pardo_avoiding_2013}{}%
Pardo SA, Cooper AB, Dulvy NK (2013) Avoiding fishy growth curves.
\emph{Methods in Ecology and Evolution} \textbf{4}, 353--360.
doi:\href{https://doi.org/10.1111/2041-210x.12020}{10.1111/2041-210x.12020}.

\leavevmode\hypertarget{ref-pardo_maximum_2016}{}%
Pardo SA, Kindsvater HK, Reynolds JD, Dulvy NK (2016) Maximum intrinsic
rate of population increase in sharks, rays, and chimaeras: The
importance of survival to maturity. \emph{Canadian Journal of Fisheries
and Aquatic Sciences} \textbf{73}, 1159--1163.
doi:\href{https://doi.org/10.1139/cjfas-2016-0069}{10.1139/cjfas-2016-0069}.

\leavevmode\hypertarget{ref-passerotti_updates_2012}{}%
Passerotti MS, Baremore IE (2012) Updates to age and growth parameters
for blacktip shark, \emph{Carcharhinus limbatus} , in the Gulf of
Mexico. SEDAR, (North Charleston, SC)

\leavevmode\hypertarget{ref-quattro_sphyrna_2013}{}%
Quattro JM, Driggers WB, Grady JM (2013) \emph{Sphyrna gilberti sp.
nov}., A new hammerhead shark (Carcharhiniformes, Sphyrnidae) from the
western Atlantic Ocean. \emph{Zootaxa} \textbf{3702}, 159--178.

\leavevmode\hypertarget{ref-rcoreteam_r:_2018}{}%
R Core Team (2018) `R: A Language and Environment for Statistical
Computing.' (R Foundation for Statistical Computing: Vienna, Austria)

\leavevmode\hypertarget{ref-romine_compensatory_2013}{}%
Romine JG, Musick JA, Johnson RA (2013) Compensatory growth of the
sandbar shark in the western North Atlantic including the Gulf of
Mexico. \emph{Marine and Coastal Fisheries} \textbf{5}, 189--199.
doi:\href{https://doi.org/10.1080/19425120.2013.793631}{10.1080/19425120.2013.793631}.

\leavevmode\hypertarget{ref-salini_northern_2007}{}%
Salini J, McAuley R, Blaber S, Buckworth R, Chidlow J, Gribble N,
Ovenden J, Peverell S, Pillans R, Stevens J, Tarca C, Walker T (2007)
Northern Australia sharks and rays: The sustainability of target and
bycatch fisheries. Phase 2. CSIRO Marine and Atmospheric Research,
Cleveland,

\leavevmode\hypertarget{ref-simpfendorfer_bright_2017}{}%
Simpfendorfer CA, Dulvy NK (2017) Bright spots of sustainable shark
fishing. \emph{Current Biology} \textbf{27}, R97--R98.
doi:\href{https://doi.org/10.1016/j.cub.2016.12.017}{10.1016/j.cub.2016.12.017}.

\leavevmode\hypertarget{ref-simpfendorfer_utilisation_1993}{}%
Simpfendorfer CA, Milward NE (1993) Utilisation of a tropical bay as a
nursery area by sharks of the families Charcharinidae and Sphyrinidae.
\emph{Environmental Biology of Fishes} \textbf{37}, 337--345.

\leavevmode\hypertarget{ref-smart_age_2015}{}%
Smart JJ, Chin A, Tobin AJ, Simpfendorfer CA, White WT (2015) Age and
growth of the common blacktip shark \emph{Carcharhinus limbatus} from
Indonesia, incorporating an improved approach to comparing regional
population growth rates. \emph{African Journal of Marine Science}
\textbf{37}, 177--188.
doi:\href{https://doi.org/10.2989/1814232X.2015.1025428}{10.2989/1814232X.2015.1025428}.

\leavevmode\hypertarget{ref-smith_intrinsic_1998}{}%
Smith SE, Au DW, Show C (1998) Intrinsic rebound potentials of 26
species of Pacific sharks. \emph{Marine and Freshwater Research}
\textbf{49}, 663--678.
doi:\href{https://doi.org/10.1071/MF97135}{10.1071/MF97135}.

\leavevmode\hypertarget{ref-smith_morphometric_2009}{}%
Smith WD, Bizzarro JJ, Richards VP, Nielsen J, Márquez-Flarias F, Shivji
MS (2009) Morphometric convergence and molecular divergence: The
taxonomic status and evolutionary history of \emph{Gymnura
crebripunctata} and \emph{Gymnura marmorata} in the eastern Pacific
Ocean. \emph{Journal of Fish Biology} \textbf{75}, 761--783.

\leavevmode\hypertarget{ref-stevens_biology_1986}{}%
Stevens JD, Wiley PD (1986) Biology of two commercially important
carcharhinid sharks from northern Australia. \emph{Marine and Freshwater
Research} \textbf{37}, 671--688.
doi:\href{https://doi.org/10.1071/MF9860671}{10.1071/MF9860671}.

\leavevmode\hypertarget{ref-taylor_size_2013}{}%
Taylor SM, Bennett MB (2013) Size, sex and seasonal patterns in the
assemblage of Carcharhiniformes in a sub-tropical bay. \emph{Journal of
Fish Biology} \textbf{82}, 228--241.

\leavevmode\hypertarget{ref-taylor_unconfounding_2009}{}%
Taylor IG, Gallucci VF (2009) Unconfounding the effects of climate and
density dependence using 60 years of data on spiny dogfish
(\emph{Squalus acanthias}). \emph{Canadian Journal of Fisheries and
Aquatic Sciences} \textbf{66}, 351--366.

\leavevmode\hypertarget{ref-then_evaluating_2015}{}%
Then AY, Hoenig JM, Hall NG, Hewitt DA (2015) Evaluating the predictive
performance of empirical estimators of natural mortality rate using
information on over 200 fish species. \emph{ICES Journal of Marine
Science: Journal du Conseil} \textbf{72}, 82--92.
doi:\href{https://doi.org/10.1093/icesjms/fsu136}{10.1093/icesjms/fsu136}.

\leavevmode\hypertarget{ref-tillett_accuracy_2012}{}%
Tillett BJ, Field IC, Bradshaw CJ, Johnson G, Buckworth RC, Meekan MG,
Ovenden JR (2012) Accuracy of species identification by fisheries
observers in a north Australian shark fishery. \emph{Fisheries Research}
\textbf{127-128}, 109--115.
doi:\href{https://doi.org/https://doi.org/10.1016/j.fishres.2012.04.007}{https://doi.org/10.1016/j.fishres.2012.04.007}.

\leavevmode\hypertarget{ref-tovar-avila_edad_2009}{}%
Tovar-Ávila J, Arenas-Fuentes V, Chiappa-Carrara X (2009) Edad y
crecimiento del tiburón puntas negras, \emph{Carcharhinus limbatus}, en
el Golfo de México. \emph{Ciencia pesquera} \textbf{17}, 48.

\leavevmode\hypertarget{ref-walker_spatial_2007}{}%
Walker TI (2007) Spatial and temporal variation in the reproductive
biology of gummy shark \emph{Mustelus antarcticus} (Chondrichthyes :
Triakidae) harvested off southern Australia. \emph{Marine and Freshwater
Research} \textbf{58}, 67--97.
doi:\href{https://doi.org/10.1071/MF06074}{10.1071/MF06074}.

\leavevmode\hypertarget{ref-walters_shark_1997}{}%
Walters CJ, Buckworth RC (1997) Shark and Spanish Mackerel stocks
assessed. \emph{Northern Territory Fisheries Industry Council
Newsletter} \textbf{8}, 14--15.

\leavevmode\hypertarget{ref-white_catch_2007}{}%
White WT (2007) Catch composition and reproductive biology of whaler
sharks (Carcharhiniformes: Carcharhinidae) caught by fisheries in
Indonesia. \emph{Journal of Fish Biology} \textbf{71}, 1512--1540.
doi:\href{https://doi.org/10.1111/j.1095-8649.2007.01623.x}{10.1111/j.1095-8649.2007.01623.x}.

\leavevmode\hypertarget{ref-white_review_2012}{}%
White WT, Last PR (2012) A review of the taxonomy of chondrichthyan
fishes: A modern perspective. \emph{Journal of Fish Biology}
\textbf{80}, 901--917.
doi:\href{https://doi.org/10.1111/j.1095-8649.2011.03192.x}{10.1111/j.1095-8649.2011.03192.x}.

\leavevmode\hypertarget{ref-whitley_new_1950}{}%
Whitley GP (1950) A new shark from north-western Australia.
\emph{Western Australian Naturalist} \textbf{2}, 100--105.

\leavevmode\hypertarget{ref-wintner_age_1996}{}%
Wintner SP, Cliff G (1996) Age and growth determination of the blacktip
shark, \emph{Carcharhinus limbatus}, from the east coast of South
Africa. \emph{Fishery Bulletin} \textbf{94}, 135--144.

\leavevmode\hypertarget{ref-xiao_relationship_2002}{}%
Xiao YS (2002) Relationship among models for yield per recruit analysis,
models for demographic analysis, and age- and time-dependent stock
assessment models. \emph{Ecological Modelling} \textbf{155}, 95--125.
doi:\href{https://doi.org/10.1016/S0304-3800(02)00028-5}{10.1016/S0304-3800(02)00028-5}.

\leavevmode\hypertarget{ref-xiao_demographic_2000}{}%
Xiao YS, Walker TI (2000) Demographic analysis of gummy shark
(\emph{Mustelus antarcticus}) and school shark (\emph{Galeorhinus
galeus}) off southern Australia by applying a generalized Lotka equation
and its dual equation. \emph{Canadian Journal of Fisheries and Aquatic
Sciences} \textbf{57}, 214--222.
doi:\href{https://doi.org/10.1139/f99-224}{10.1139/f99-224}.


\end{document}
